\documentclass[letterpaper, 12pt]{article}

\usepackage{caption}
\usepackage{float}
\usepackage[T1]{fontenc}
\usepackage[lmargin=1 in, rmargin=1 in, tmargin=1 in, bmargin=1 in]{geometry}
\usepackage{graphicx}
\usepackage{hyperref}
\usepackage{times}
\usepackage{xcolor}

\begin{document}

\textbf{Associate editor}

\bigskip

\textit{In addition to the reviewer's comments, the following references should be cited:
\begin{itemize}
\item L48. Study of LFE has been continuing for about two decades, and it may be difficult to cite all the important previous studies. The review of Beroza and Ide (2011, Annual Rev. Earth Planet. Sci.) covers the first decade.
\item L53. Tremor can be explained as a swarm of LFEs (Shelly et al., 2007, Nature).
\item L56. The source of the tremor and the LFEs is located on the plate boundary (Shelly et al., 2006, Nature; Bostock et al., 2012, G-cubed). Or as a summary by Audet and Kim (2016, Tectonophysics).
\item L58. In addition to the currently cited Ide et al. (2007, GRL), Bostock et al. (2012, G-cubed) and Royer and Bostock (2014, EPSL) are also appropriate for the mechanism study in Cascadia.
\item L63. Obara (2002, Science) should obviously be cited, but is inappropriate as a basis for a correlation study between tectonic tremor and slow slip. Obara et al. (2004, GRL) is better.
\end{itemize}}

\bigskip

We have added these references in the text.

\bigskip

\textbf{Reviewer 1}

\bigskip

\textit{Figure 1: Depth contour of subducting plate and surface traces of the San Andreas fault zones (SAF) should be in the figure for easier understanding because authors mentioned about LFE families are on SAF and in the subduction zones (Lines 176 - 183). It should be also helpful to include a figure of depth cross section for LFE hypocenters.}

\bigskip

\textit{Lines 73 - 74: Bostock (2015, https://doi.org/10.1002/2015JB012195) provides 10-year catalog of LFES at the Vancouver Island.}

\bigskip

We added the reference in the text.

\bigskip

\textit{Lines 75 - 76: Japan Meteorological Agency (JMA) provides LFE catalog in Japan since 2004. Kato and Nakagawa (2020, https://doi.org/10.1186/s40623-020-01257-4) also provides LFE catalogs in Japan between 2004 - 2015.}

\bigskip

We added the references in the text.

\bigskip

\textit{Line 138: Why did authors used such a long (1 min) time window for matched filter analysis. I think it is common to use time windows of 5 - 10 seconds around the arrival time at each seismic station (e.g., Shelly et al., 2007, https://doi.org/10.1038/nature05666; Kato and Nakagawa, 2020, https://doi.org/10.1186/s40623-020-01257-4).}

\bigskip

The templates from Plourde et al. (2015) have a low signal-to-noise ratio and generally do not show clear P- and S-wave arrivals. Moreover, the wave arrivals can be located close to the beginning or close to the end of the time window, depending on the LFE family considered. We have chosen to keep the whole one-minute time window to be sure to include both the P- and the S-wave arrivals. The templates obtained using LFE detections from the 2-year-long catalog show clearer P- and S-wave arrivals and we have reduced the length of the time window to 30 seconds on average, depending on the LFE family considered. We have remade the templates using only LFEs that were present in both catalogs (FAME and permanent networks). The wave arrivals are now clearer and, to extend farther in time our catalog, we could reduce the size of the time window by using a different starting time for each station, depending on whether it is close or farther away from the LFE source location. 

\bigskip

\textit{Figure 2: It is helpful to show the value of the detection threshold in this figure.}

\bigskip

\textit{Lines 165 - 175: How second thresholds (CC values vs stations numbers) were chosen for each seismic station? If it is done manually, please provide a list of the threshold as the supplementary table. In addition, how many events did this selection step remove from the catalog? (A figure comparing before and after the selection step will be useful.)}

\bigskip

\textit{Lines 204 - 207: How much are two catalogs (the FAME catalog and the network catalog) different originally? As the station density of FAME stations seems to be equivalent to that of permanent stations, I expect two catalogs do not show significant differences. If there are severe differences between two catalogs, detection process and/or event selection procedure needs to be reconsidered. In addition, I think it is also good to show event catalog detected using both FAME stations and permanent stations. Comparing "FAME + network" catalog and the network catalog at the same period, we can evaluate how much the resolution of LFE detection only with the permanent stations becomes worse than that with additional FAME stations. This is important information to evaluate completeness of the LFE catalog in the periods without FAME stations.}

\bigskip

\textit{Line 209: "two thresholds" for what?}

\bigskip

\textit{Line 215: It will be helpful to visually show temporal changes of the number of available seismic stations.}

\bigskip

\textit{Line 249: Surface wave triggering of LFEs at the Nankai trough was reported in Miyazawa et al. (2008, https://doi.org/10.1186/BF03352858) and references therein before Han et al. (2014).}

\bigskip

We added the reference in the text.

\bigskip

\textit{Lines 271 - 273: Is there no evidence for dynamic triggering as well?}

\bigskip

\textit{Lines 290 - 294: I think the vertical axis of Figure 9 should be average recurrence interval rather than the number of events.}

\bigskip

\textit{Line 296: Tidal triggering of tremors was reported in many studies before Houston (2015), including Shelly et al. (2007, https://doi.org/10.1029/2007GC001640) and Houston et al. (2011, https://doi.org/10.1038/ngeo1157).}

\bigskip

We added these references.

\bigskip

Line 306: Which fault plane did authors assume to calculate tidal stress?

Lines 323 - 324: Is it possible that this negative sensitivity for family A is due to incorrect fault geometry? As family A is located at the southern edge of the subduction zone, fault geometry could be complex, which might be different from the assumed one.

\bigskip

\textbf{Reviewer 2}

\bigskip

\textit{The southern-most LFE family is very interesting! I agree with your assessments that the tidal stress modulation and rapid recurrence intervals suggest this family is on a different fault. With that in mind, I find the lack of further discussion about it unsastisfying. Did you try to re-stack and re-locate this family? It would be great if you could. Even if you conclude that you can’t get a reliable depth estimate, I would suggest it’s worth showing the template waveforms and discussing a few potential source regions, even if it’s highly speculative.}

\bigskip

Line 290 / Figure 9: I’m a bit confused here. The number of LFE events shown on Figure 9 (< 50) are much too small to be total numbers of events, as made clear by the daily event counts in Figure 8 … Are these LFE numbers normalised somehow? Please clarify this. 

Minor points

Line 22: what does “other” refer to? Not associated with tremor detections? 

I think the plain-language summary could be more “plain”. Even words/phrases like “episodically”, “recurrence intervals”, “tidal stress sensitivity” could be expressed in 


simpler/more-common words. Things like “tectonic tremor”, “slow slip” are certainly not familiar to most people, so that’s not plain language. 

Line 41: “…recurrence intervals smaller…” → shorter recurrence intervals 

Line 61: “…in a bursty manner.” Maybe you can follow this up with a more explicit description of what “bursty” means…something like “tens to hundreds of events in short period, often lasting a few hours”? 

Line 92: “…were located above the plate boundary.” We found that they were shallower than the McCrory model, but we actually assumed that this meant the McCrory model was biased deep. A minor rephrase would be good here (even though we might have just been wrong). 

Figure 1: it would be nice to have the trench, the San Andreas Fault, (+ the Maacama and Bucknell Creek faults?) and/or plate interface contours on this map 

Line 139: “All these preprocessing operations are done with the Python package obspy”. I think it is fine to leave this to the acknowledgements if you prefer to shorten here. 

\bigskip

\textit{Line 151: maybe a reference to that Shelly (2007, Nature) paper is warranted here for the somewhat standard 8*MAD threshold}

We added the reference.

\bigskip

Line 161: does the 74\% apply to the total number of LFEs (all families)? Or still the 61 families of the previous sentence? 

Line 173: “The threshold is thus different for each LFE family”. I’m not sure what you mean here. Is the threshold constant for each family, or does it vary as stations become active/inactive? Is the threshold consistent for a given number of active stations (on that day), or does it vary between templates? Please try to clarify. 

Line 180: “Additionally, one LFE family located on the southern end of the subduction zone has also shorter recurrence intervals than families located farther north, and behaves more similarly to the strike-slip fault families.” What locations are you using here? Also, maybe for clarity you can put (40.1⁰ N), or whatever the precise latitude is after “…on the southern end”. It may be good to specify the depth here too, since it may not be surprising to see frequent bursts if the family is at 45 km depth. 

Line 223: I think it would be nice to quantify the number of shared (between FAME and network catalogs) events, for each family here instead of just saying “…most events seem to be present…” 

Figure 7: Did the Boyarko et al. study include 2004? I found in your discussion that it does not, but I think it’s worth repeating this information in the figure caption. Also, the font size is quite tiny here, it would be nice to ensure that there is no text below size 6-7 on any of the figures. 

Figure 9: Does the “upper limit of tremor” mean the westward limit? 

Line 309: Can you provide references to justify these coefficient values (0.1, 0.5)? 

Line 312: “We also computed what would be the expected number of LFEs occurring if the tidal stress changes have no influence on LFE activity.” It’s not clear to me why this (the black lines in Figure 10) varies so dramatically between families, could you explain a bit further? 

Supplementary Material: it might be worth repeating the acronym explanations (IRIS, NDEDC) as well as the links to the data sources in this document 

\end{document}