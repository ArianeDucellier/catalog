\documentclass[letterpaper, 12pt]{article}

\usepackage{caption}
\usepackage{float}
\usepackage[T1]{fontenc}
\usepackage[lmargin=1 in, rmargin=1 in, tmargin=1 in, bmargin=1 in]{geometry}
\usepackage{graphicx}
\usepackage{hyperref}
\usepackage{times}
\usepackage{xcolor}

\begin{document}

\textbf{Reviewer 2}

\bigskip

\textit{Figure 1 and Figure 9 captions: I think it would be nice if the Figure 1 had a legend with the different shapes (triangle, pentagon, etc.) for the different group numbers and a corresponding explanation in the text. Then the notes in later figure captions specifying the shapes could be removed. It is a bit distracting to see the shapes without explanation when you first see Figure 1.}

\bigskip

We added the legend with the different shapes in the caption of Figure 1 and removed it from the captions in the other figures.

\bigskip

\textit{Line 194: I do not think MAD has been defined yet?}

\bigskip

We define MAD at line 161. We added the acronym, which was missing.

\bigskip

\textit{Figure 3 caption: It might be useful to say something like "LFE and tremor detections during the FAME experiment as a function of time..." or something that distinguishes the difference between this figure and Figure 6 . A legend with dot sizes and the corresponding number of LFEs would also be useful.}

\bigskip

We specified the data used to detect the LFEs shown in Figure 3 and 6. We added dot sizes and the corresponding number of LFEs in the legend.

\bigskip

\textit{Line 283: It is known than , that*}

\bigskip

We corrected this sentence.

\bigskip

\textit{Line 328: I think you are defining an LFE cluster in this section, not "assuming" these characteristics about clusters.}

\bigskip

We corrected this sentence and added definitions of LFE cluster and LFE episode earlier in the text.

\end{document}